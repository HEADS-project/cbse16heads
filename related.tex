\section{Related Work}

Some attempts of creating an Esperanto of the programming languages have emerged, aiming at replacing the need for using multiple programming languages. For example, Haxe~\cite{dasnois2011haxe} is a programming language that cross-compile to most of the popular programming languages (Java, JavaScript, C++ and others). A significant development effort in Haxe, beyond the development of the language and compilers, is put in the development of a standard library (mostly wrapping the standard libraries of the targeted languages). This standard library needs to be embedded at runtime with a too large overhead to run on the most constrained platforms our approach is able to target. 

In the CBSE domain, some component platforms are available for different languages. For example, Fractal~\cite{bruneton2006fractal} is implemented in Java and C. However, no significant effort has been put to make the different platforms to interoperate, those different implementations being isolated rather than being a versatile Fractal. The HEADS approach facilitates, both at design-time and runtime the integration and interoperability of components expressed in different languages.

In the modeling domain, using a multi-viewpoints approaches proved to be profitable. Schmidt et al ~\cite{DBLP:conf/middleware/GokhaleSLNW03} use a modeling approach to generate Distributed Real-time and Embedded Component Middleware and Applications. More recently, in  ~\cite{5753608}, Dabholkar et al highlights the benefits of providing a Generative Middleware Specialization Process for Distributed Real-Time and Embedded Systems.  The HEADS approach  mainly focuses in enabling a simple integrating of some viewpoints with existing programming language to simplify the complex integration of modern systems. 

In the system Engineering community, the Eclipse Polarsys Capella project~\footnote{https://www.polarsys.org/capella/} propose a native support for viewpoint extensions, allowing to extend and/or specialize the core environment to address particular engineering concerns (performance, operating safety, security, cost, weight, product line, etc.), combined with the possibility to carry out multi-criteria analysis of target architectures to help find the best trade-offs~\cite{voirin2013arcadia}. In that direction, the HEADS approach can be compared to the polarsys project with a focus on the integration with existing programming language.

In the Mobile domain, Cepa et al~\cite{1385821} present MobCon a top-down generative approach to create Middleware for Java Mobile Applications, which shows that generation techniques can be effectively used to develop mobile application. In the same direction, Cassou et al~\cite{Cassou:2011:LSA:1985793.1985852} presents a generative approaches based on software architecture model, associated with verification strategies, to create pervasive applications. Contrary to this work, we do not provide a unique abstract viewpoint to design the system. The HEADS approach provides a specific viewpoint for each stakeholder or each software building stage.
